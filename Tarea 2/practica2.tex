\documentclass{article}
\usepackage[spanish]{babel}
\usepackage[numbers,sort&compress]{natbib}
\usepackage[T1]{fontenc}
\usepackage[utf8]{inputenc}
\usepackage{graphicx}
\usepackage{url}
\usepackage[numbers,sort&compress]{natbib}


\title {Autómata celular}
\author{Julio Garc\'ia}


\pagestyle{empty}

\begin{document}

\maketitle



\section{Objetivo}

Basados en la definición de autómata celular presentado en \citet{e1}
 en dos dimensiones, especialmente el famoso juego de la vida \citet{e2}
 , se desea diseñar y ejecutar un experimento para determinar el mayor tiempo continua de vida en una celda en una malla de 20 por 20 celdas hasta que se mueran todas o que se hayan cumplidos una cantidad de iteraciones definida (50). En este experimento se varía la probabilidad de 0.1 a 0.9 en pasos de 0.1 (En el ejemplo se distribuyen uniformemente al azar, por lo cual la probabilidad es 0.5) 


\section{Metodolog\'ia}
Una matriz boleana representa el estado de un automata, es decir, una celda se representa como una entrada en una matriz, si su valor es uno, indica que esta viva, de lo contrario si vale cero indica que esta muerta. En \citet{p2}
 se indica el proceso de supervivencia de una matriz (autómata celular) de una iteración a otra, básicamente se centra en la suma de los vecinos de una celda, deberá ser exactamente tres, si esto ocurre, la celda sobrevivirá, en la siguiente iteración. En base a lo definido anteriormente, se supone lo siguiente:
-El tiempo de vida de una celda, dependerá si entre la iteración antecesora y la iteración sucesora se mantiene viva dicha celda, es decir, de tener un valor uno, continua con otro valor. Por lo cual, el valor se acumula entre iteraciones, en caso de morir la celda, su tiempo de vida se acaba y vuelve hacer cero.
-El autómata celular inicial, no es distribuida uniformemente, su probabilidad de cada celda varía entre 0.1 a 0.9. Este supuesto es muy importante, ya que la semilla (autómata celular inicial) es la propagante inicial para que existan más sobrevivientes y la vida de las celdas sea más largas.


\newpage
\section{Resultados}
El experimento fue codificado en el lenguaje de programación Python, se realizaon 50 iteraciones con una automata inicial con probabilidad entre 0.1 y 0.9. Se modifico el código original al generar la automata inicial, se guardo el valor máximo de vida entre todas las iteraciones, así como la vida entre cada iteración. 

\begin{tabular}{|c|c|c|}
	\hline
	Iteracion & Suma de vivos & Máximo de dos iteraciones \\
	\hline
	0 & 65 & 0 \\
	\hline
	1 & 41 & 1 \\
	\hline
	2 & 28 & 1 \\
	\hline
	3 & 21 & 1 \\
	\hline
	4 & 16 & 2 \\
	\hline
	5 & 7 & 1 \\
	\hline
	6 & 4 & 1 \\
	\hline
	7 & 4 & 1 \\
	\hline
	8 & 4 & 1 \\
	\hline
	9 & 4 & 1 \\
	\hline
	10 & 4 & 1 \\
	\hline
	11 & 4 & 1 \\
	\hline
	12 & 4 & 1 \\
	\hline
	13 & 4 & 1 \\
	\hline
	14 & 4 & 1 \\
	\hline
	15 & 4 & 1 \\
	\hline
	16 & 4 & 1 \\
	\hline
	17 & 4 & 1 \\
	\hline
	18 & 4 & 1 \\
	\hline
	19 & 4 & 1 \\
	\hline
	20 & 4 & 1 \\
	\hline
	21 & 4 & 1 \\
	\hline
	22 & 4 & 1 \\
	\hline
	23 & 4 & 1 \\
	\hline
	24 & 4 & 1 \\
	\hline
	25 & 4 & 1 \\
	\hline
	26 & 4 & 1 \\
	\hline
	27 & 4 & 1 \\
	\hline
	28 & 4 & 1 \\
	\hline
	29 & 4 & 1 \\
	\hline
	30 & 4 & 1 \\
	\hline
	31 & 4 & 1 \\
	\hline
	32 & 4 & 1 \\
	\hline
	33 & 4 & 1 \\
	\hline
	34 & 4 & 1 \\
	\hline
	35 & 4 & 1 \\
	\hline
	36 & 4 & 1 \\
	\hline
	37 & 4 & 1 \\
	\hline
	38 & 4 & 1 \\
	\hline
	39 & 4 & 1 \\
	\hline
	40 & 4 & 1 \\
	\hline
	41 & 4 & 1 \\
	\hline
	42 & 4 & 1 \\
	\hline
	43 & 4 & 1 \\
	\hline
	44 & 4 & 1 \\
	\hline
	45 & 4 & 1 \\
	\hline
	46 & 4 & 1 \\
	\hline
	47 & 4 & 1 \\
	\hline
	48 & 4 & 1 \\
	\hline
	49 & 4 & 1 \\
	\hline
	50 & 4 & 1 \\
	\hline
\end{tabular}


\section{Conclusi\'on}
La vida máxima en este experimento es igual a dos iteraciones continuas para una celda. Además, podemos ver que hay cuatro celdas que se mantienen vivas desde la iteración 6 a la 50, es importante mencionar que las celdas no son las mismas, sin embargo, si están muy cercanas.\citet{p_2}


\bibliography{Biblio}
\bibliographystyle{plainnat}

\end{document}