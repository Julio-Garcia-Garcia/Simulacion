\documentclass{article}
\usepackage[spanish]{babel}
\usepackage[numbers,sort&compress]{natbib}
\usepackage[T1]{fontenc}
\usepackage[utf8]{inputenc}
\usepackage{graphicx}
\usepackage{url}
\usepackage[numbers,sort&compress]{natbib}

\title { Movimiento Browniano}
\author{Julio Garc\'ia}

\begin{document}

\maketitle


\section{Introducci\'on}

El fenómeno físico del movimiento browniano hace referencia al desplazamiento errático de las partículas diminutas que se encuentran inmersas en alguna sustancia. El descubrimiento de este fenómeno fue protagonizado a principios del siglo XlX por un botánico y médico escocés, Robert Brown.

\section{Objetivo}
Examinar de manera sistemática los efectos de la dimensión en el tiempo de regreso al origen del movimiento Browniano.

\section{Metodolog\'ia}
Se uso el  lenguaje de programación python, para encontrar	los efectos de la dimensión en el tiempo de regreso al origen del movimiento Browniano para dimensiones 1 a 8 en incrementos lineales de uno, variando el número de pasos de la caminata como potencias de dos con exponente de 5 a 10 en incrementos lineales de uno, con 50 repeticiones del experimento para cada combinación. Y el mínimo, promedio y máximo del tiempo de regreso por cada dimensión junto con el porcentaje de caminatas que nunca regresaron.

\newpage
\section{Resultados}
Con simulcion en python, en base  la cantidad de pasos antes de llegar al origen obtuvimos: \\ \\


\begin{tabular}{| c | c | c | c |}
	\hline
	Dimension & Mínimo & Máximo & Promedio\\ \hline
	1 & 0 & 176 & 7.368 \\
	2 & 0 & 432 & 14.904\\
	3 & 0 & 326 & 9.152\\
	4 & 0 & 224 & 9.496\\
	5 & 0 & 178 & 9.8\\
	6 & 0 & 284 & 8.904\\
	7 & 0 & 194 & 8.424\\
	8 & 0 & 200 & 9.6\\ \hline

	
\end{tabular}

\section{Conclusi\'on}
Gracias al movimiento de una particula de manera aleatoria, sirve para la modelación matemática discretisando el tiempo.

\bibliography{biblio}
\bibliographystyle{plainnat}

\end{document}